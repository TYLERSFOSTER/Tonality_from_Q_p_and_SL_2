\documentclass[letterpaper,11pt, reqno]{amsart}

\usepackage{amsfonts, amsthm, amssymb, amsmath, stmaryrd}
\usepackage{bbm}
\usepackage{mathrsfs,array}
\usepackage{eucal,fullpage,times,color,enumerate,accents,mathtools}
\usepackage{url}
\usepackage{scalerel,stackengine}
\usepackage{dotlessj}
\usepackage{cancel}
\usepackage{pgfplots}
\usepackage{colortbl,hhline}
\usepackage{tikz}
\usetikzlibrary{decorations.pathmorphing}
\usetikzlibrary{patterns}
\usetikzlibrary{arrows}
\usepackage[all]{xy}
\usepackage{soul}
\usepackage{xcolor}
\usepackage{etoolbox}
\usepackage{pifont}
\usetikzlibrary{fadings}
\usetikzlibrary{calc}
\usepackage{listings}
\usepackage{wasysym}
\tikzfading[name=fade out,
  inner color=transparent!0, outer color=transparent!100]
\tikzfading[name=fade right,
  left color=transparent!0, right color=transparent!100]
\tikzfading[name=fade left,
  right color=transparent!0, left color=transparent!100]
\tikzfading[name=fade mid,
  left color = transparent!100, right color = transparent!100, middle color=transparent!0]
\usepackage{ifthen}
\tikzset{
  laser beam action/.style={
    line width=\pgflinewidth+1.4pt,draw opacity=.095,draw=#1,
  },
  laser beam recurs/.code 2 args={%
    \pgfmathtruncatemacro{\level}{#1-1}%
    \ifthenelse{\equal{\level}{0}}%
    {\tikzset{preaction={laser beam action=#2}}}%
    {\tikzset{preaction={laser beam action=#2,laser beam recurs={\level}{#2}}}}
  },
  laser beam/.style={preaction={laser beam recurs={20}{#1}},draw opacity=1,draw=#1},
}
\usepackage{graphics}
\usepackage{graphicx}
\usepackage[export]{adjustbox}
\usepackage[curve]{xypic}
\usepackage{bm}
\usetikzlibrary{calc}
\usepackage[font=small,labelfont=bf]{caption}
\usepackage{stackengine,scalerel,graphicx}
\savestack\UAtextstyle{\stackon[-2.7pt]{$\rule[2.3pt]{4pt}{.35pt}$}{\scalebox{-1}{$U$}}}
\def\UA{\scalerel*{\UAtextstyle}{X}}
\usepackage{arydshln}



\definecolor{navy}{rgb}{0,0,.65}

%This reverse-links the references in the paper. Useful for large papers.
\usepackage[colorlinks]{hyperref}
\hypersetup{colorlinks=true,urlcolor=teal,linkcolor=navy,citecolor=navy}

\makeatletter
\def\@tocline#1#2#3#4#5#6#7{\relax
  \ifnum #1>\c@tocdepth % then omit
  \else
    \par \addpenalty\@secpenalty\addvspace{#2}%
    \begingroup \hyphenpenalty\@M
    \@ifempty{#4}{%
      \@tempdima\csname r@tocindent\number#1\endcsname\relax
    }{%
      \@tempdima#4\relax
    }%
    \parindent\z@ \leftskip#3\relax \advance\leftskip\@tempdima\relax
    \rightskip\@pnumwidth plus4em \parfillskip-\@pnumwidth
    #5\leavevmode\hskip-\@tempdima
      \ifcase #1
       \or\or \hskip 1em \or \hskip 2em \else \hskip 3em \fi%
      #6\nobreak\relax
    \dotfill\hbox to\@pnumwidth{\@tocpagenum{#7}}\par
    \nobreak
    \endgroup
  \fi}
\makeatother

\renewcommand{\familydefault}{ppl}
\setlength{\marginparwidth}{1in}
\setlength{\marginparsep}{0in}
\setlength{\marginparpush}{0.1in}
\setlength{\topmargin}{0in}
\setlength{\headheight}{0pt}
\setlength{\headsep}{0pt}
\setlength{\footskip}{.3in}
\setlength{\textheight}{9.0in}
\setlength{\textwidth}{6.25in}
\setlength{\parskip}{0pt}

%\newtheorem{theorem}{Theorem}[section]
\newtheorem{idea}{Musical proto-idea}[]
\renewcommand*{\theidea}{\arabic{idea}}
\newtheorem{ideaB}{Simple musical idea}[]
\renewcommand*{\theideaB}{\arabic{ideaB}}
\newtheorem{composition}{Composition}[]
\renewcommand*{\thecomposition}{\Alph{composition}}
\newtheorem{theorem}{Theorem}[subsection]
\newtheorem{monodromy theorem}{Monodromy Theorem}[subsection]
\newtheorem{corollary}[theorem]{Corollary}
\newtheorem{hypothesis}[theorem]{Hypothesis}
\newtheorem{wild conjecture}[theorem]{Wild Conjecture}
\newtheorem{claim}[theorem]{Claim}
\newtheorem{lemma}[theorem]{Lemma}
\newtheorem{proposition}[theorem]{Proposition}
\newtheorem{definition}[theorem]{Definition}
\newtheorem{warning}[theorem]{Warning}
\newtheorem{research objectives}{Research objectives}[subsection]
\newtheorem{questions}{Question}
\newtheorem{question}[theorem]{Question}
\newtheorem{research question}[theorem]{Research questions}
\newtheorem{answer}[theorem]{Answer}
\newtheorem{aside question}[theorem]{Aside question}
\newtheorem{exercise}[theorem]{Exercise}
\newtheorem{sketch}[theorem]{Sketch}
\newtheorem{aside}[theorem]{Aside}
\newtheorem{problem}[theorem]{Problem}
\newtheorem{conjecture}[theorem]{Conjecture}
\newtheorem{assumption}[theorem]{Assumption}
\newtheorem{construction}[theorem]{Construction}
\newtheorem{example}[theorem]{Example}
\newtheorem{examples}[theorem]{Examples}
\newtheorem{audio example}[theorem]{\loudspeaker[3] Example}
\newtheorem{quasi-theorem}[theorem]{Quasi-Theorem}
\newtheorem{prop/def}[theorem]{Proposition/Definition}
\newtheorem{blank remark}[theorem]{}
\newtheorem{ssubsection}[theorem]{}
\newtheorem{terminology and comment}[theorem]{Terminology and comment}
\newtheorem{fact}[theorem]{Fact}
\newtheorem{computation}[theorem]{Computation}
\newtheorem{observation}[theorem]{Observation}
\newtheorem{algorithm}[theorem]{Algorithm}
\newtheorem{setup}[theorem]{Setup}
\newtheorem{purity hypothesis}[theorem]{Purity hypothesis}
\newtheorem{corollary of the purity hypothesis}[theorem]{Corollary of the purity hypothesis}

%Theorem indexed with letters A, B, C, ...
\newtheorem{Th}{Theorem}[]
\renewcommand*{\theTh}{\Alph{Th}}

\newtheorem{rem1}[theorem]{Remark}
\newenvironment{remark}{\begin{rem1}\em}{\end{rem1}}

\newtheorem{not1}[theorem]{Notation}
\newenvironment{notation}{\begin{not1}\em}{\end{not1}}

%% Math Blackboard
\newcommand{\A}{{\mathbb{A}}}           
\newcommand{\CC} {{\mathbb C}}       
\newcommand{\DD} {{\mathbb D}}
\newcommand{\EE}{\mathbb{E}}
\newcommand{\GG}{\mathbb{G}}
\newcommand{\LL}{\mathbb{L}}
\newcommand{\NN} {{\mathbb N}}		
\newcommand{\PP}{\mathbb{P}}         
\newcommand{\QQ} {{\mathbb Q}}		
\newcommand{\RR} {{\mathbb R}}		
\newcommand{\Circ} {{\mathbb S}}		
\newcommand{\ZZ} {{\mathbb Z}}		
\newcommand{\TT} {{\mathbb T}}	
\newcommand{\FF}{{\mathbb F}}

%%Presuperscript
\def\presuper#1#2%
  {\mathop{}%
   \mathopen{\vphantom{#2}}^{#1}%
   \kern-\scriptspace%
   #2}
	
\newcommand{\BC}{\text{BC}}

\DeclareMathOperator{\Aut}{Aut}
\DeclareMathOperator{\Gal}{Gal}
\DeclareMathOperator{\Circpec}{Spec_{\ \!}}
\DeclareMathOperator{\Split}{split}
\DeclareMathOperator{\Div}{div}
\DeclareMathOperator{\ord}{ord_{\ \!}}

\newcommand{\model}[1]{{\slantbox[.5]{$\mathcal{#1}$}\ }}

\newcommand{\lra}{{\longrightarrow}}
\DeclareMathOperator{\Def}{\overset{{}_{\text{def}}}{=}}

% slant box
\newsavebox{\foobox}
\newcommand{\slantbox}[2][.5]
  {%
    \mbox
      {%
        \sbox{\foobox}{#2}%
        \hskip\wd\foobox
        \pdfsave
        \pdfsetmatrix{1 0 #1 1}%
        \llap{\usebox{\foobox}}%
        \pdfrestore
      }%
  }

%%Prettier monomorphism and epimorphism arrows
\newcommand{\mono}{\!\xymatrix{{}\ar@{^{(}->}[r]&{}}\!}
\newcommand{\epi}{\!\xymatrix{{}\ar@{->>}[r]&{}}\!}
\newcommand{\rat}{\!\xymatrix{{}\ar@{-->}[r]&{}}\!}

%%Young diagram
\newcommand{\young}{\scalebox{.7}{$\pmb{\square\!\square}${\larger\larger $\pmb{\cdot\!\cdot\!\cdot}$}$\pmb{\square}$}}

%smaller subscript closed field
\newcommand{\lilF}{\mbox{{\smaller\smaller\smaller\smaller\smaller $\overline{\FF_{\!q}}$}}}

\newcommand{\iso}{\cong}
\newcommand{\disc}{\text{disc}}

% Tyler comments
\newcommand{\tyler}[1]{{\color{red} [#1\ \ \textemdash Tyler]}}

% Some slanted letters
\newcommand{\TP}{\slantbox[.3]{$\mathcal{TP}$}}

%Left action
\newcommand{\lact}{\ \raisebox{8pt}{\rotatebox{-90}{$\circlearrowright$}}\ }

%Left quotient
\newcommand{\lquot}[2]{\raisebox{-1.5pt}{$#1$}\big\backslash\raisebox{1.5pt}{$#2$}}

%Right quotient
\newcommand{\rquot}[2]{\raisebox{1.5pt}{$#1$}/\raisebox{-1.5pt}{$#2$}}

%importantmatrix
\newcommand{\MM}{\big(\begin{smallmatrix}0 & -1\\ 1 & -1\end{smallmatrix}\big)}

%compactified spec
\newcommand{\SpecZN}[1]{\overline{\text{Spec}_{\ \!}\ZZ}{}^{(#1)}}
\newcommand{\SpecZ}{\overline{\text{Spec}_{\ \!}\ZZ}}

%nice sep
\newcommand{\sep}{\textsf{sep}}

%nice res
\newcommand{\res}{\text{res}^{\eta}_{s}}

%nice sp
\newcommand{\spe}{\bold{Sp}}

%nice Sh
\newcommand{\Sh}{\bold{Sh}}

%nice et
\newcommand{\et}{\text{\'et}}

%nice M_g bar
\newcommand{\Mg}{{\ \ \overline{\!\!\mathscr{M}_{g}}}}
\newcommand{\Mgmbar}{\overline{M_{g}{\!\!}^{\text{{\smaller\smaller\smaller\smaller\smaller $\ (m)$}}}}}
\newcommand{\Mgm}{M_{g}{\!\!}^{\text{{\smaller\smaller\smaller\smaller\smaller $\ (m)$}}}}


\stackMath
\newcommand\reallywidehat[1]{%
\savestack{\tmpbox}{\stretchto{%
  \scaleto{%
    \scalerel*[\widthof{\ensuremath{#1}}]{\kern-.6pt\bigwedge\kern-.6pt}%
    {\rule[-\textheight/2]{1ex}{\textheight}}%WIDTH-LIMITED BIG WEDGE
  }{\textheight}% 
}{0.5ex}}%
\stackon[1pt]{#1}{\tmpbox}%
}

%changed footnote style
\renewcommand{\thefootnote}{[\arabic{footnote}]}

\numberwithin{equation}{theorem}

%
\newcounter{totfigures}

\providecommand\totfig{} 

\makeatletter
\AtEndDocument{%
  \addtocounter{totfigures}{\value{figure}}%
  \immediate\write\@mainaux{%
    \string\gdef\string\totfig{\number\value{totfigures}}%
  }%
}
\makeatother

\pretocmd{\chapter}{\addtocounter{totfigures}{\value{figure}}\setcounter{figure}{0}}{}{}


\newcommand{\midarrow}{\tikz \draw[-triangle 90] (0,0) -- +(.1,0);}



\usepackage{epigraph}
\setlength\epigraphwidth{.8\textwidth}
\setlength\epigraphrule{0pt}




\definecolor{codegreen}{rgb}{0,0.6,0}
\definecolor{codegray}{rgb}{0.5,0.5,0.5}
\definecolor{codepurple}{rgb}{0.58,0,0.82}
\definecolor{backcolour}{rgb}{0.95,0.95,0.92}

\lstdefinestyle{mystyle}{
    backgroundcolor=\color{backcolour},   
    commentstyle=\color{codegreen},
    keywordstyle=\color{magenta},
    numberstyle=\tiny\color{codegray},
    stringstyle=\color{codepurple},
    basicstyle=\ttfamily\footnotesize,
    breakatwhitespace=false,         
    breaklines=true,                 
    captionpos=b,                    
    keepspaces=true,                 
    numbers=left,                    
    numbersep=5pt,                  
    showspaces=false,                
    showstringspaces=false,
    showtabs=false,                  
    tabsize=2
}

\lstset{style=mystyle}





\newcommand{\ExternalLink}{%
    \tikz[x=1.2ex, y=1.2ex, baseline=-0.05ex]{% 
        \begin{scope}[x=1ex, y=1ex]
            \clip (-0.1,-0.1) 
                --++ (-0, 1.2) 
                --++ (0.6, 0) 
                --++ (0, -0.6) 
                --++ (0.6, 0) 
                --++ (0, -1);
            \path[draw, 
                line width = 0.5, 
                rounded corners=0.5] 
                (0,0) rectangle (1,1);
        \end{scope}
        \path[draw, line width = 0.5] (0.5, 0.5) 
            -- (1, 1);
        \path[draw, line width = 0.5] (0.6, 1) 
            -- (1, 1) -- (1, 0.6);
        }
    }


\usepackage{caption}
\captionsetup{font=footnotesize}

\usepackage{graphicx}
\newcommand\vcent[1]{\vcenter{\hbox{#1}}}
\newcommand\loudspeaker[1][3]{\ensuremath{\vcent{\rule{.6ex}{.6ex}}\kern-.5ex%
  \vcent{\scalebox{.6}[1]{\rotatebox[origin=center]{90}{$\blacktriangle$}}}%
  \ifnum#1>0\relax\kern.1ex\vcent{\scalebox{.4}{)}}\ifnum#1>1\relax\kern-.1ex%
  \vcent{\scalebox{.55}{)}}\ifnum#1>2\relax\kern-.15ex\vcent{\scalebox{.7}{)}}%
  \fi\fi\fi}%
}



\newcommand\modulo[2]{\@tempcnta=#1
        \divide\@tempcnta by #2
        \multiply\@tempcnta by #2
        \multiply\@tempcnta by -1
        \advance\@tempcnta by #1\relax
        \the\@tempcnta}
\makeatother


\makeatletter
\newcommand*{\doublerightarrow}[2]{\mathrel{
  \settowidth{\@tempdima}{$\scriptstyle#1$}
  \settowidth{\@tempdimb}{$\scriptstyle#2$}
  \ifdim\@tempdimb>\@tempdima \@tempdima=\@tempdimb\fi
  \mathop{\vcenter{
    \offinterlineskip\ialign{\hbox to\dimexpr\@tempdima+1em{##}\cr
    \rightarrowfill\cr\noalign{\kern.5ex}
    \rightarrowfill\cr}}}\limits^{\!#1}_{\!#2}}}
\newcommand*{\triplerightarrow}[1]{\mathrel{
  \settowidth{\@tempdima}{$\scriptstyle#1$}
  \mathop{\vcenter{
    \offinterlineskip\ialign{\hbox to\dimexpr\@tempdima+1em{##}\cr
    \rightarrowfill\cr\noalign{\kern.5ex}
    \rightarrowfill\cr\noalign{\kern.5ex}
    \rightarrowfill\cr}}}\limits^{\!#1}}}
\makeatother




%%%%%%%%%%%%%%%%%%%%%%%%%%%%%%%%%%%%%%
%%%%%%%%%%%%%%%%%%%%%%%%%%%%%%%%%%%%%%

\setcounter{tocdepth}{2}


\title{{\smaller\smaller\smaller\smaller\it Notes on}\\ \ \\ Tonal Theories\\ {\smaller\smaller\smaller\it coming from} Representations of\\ Algebraic Groups {\smaller\smaller\smaller\it other than} $\text{GL}_{\pmb{1}}\pmb{(\CC)}$\\ \ \\ \ {\smaller\smaller\smaller\smaller\it \textemdash\ In progess\ \textemdash}}
\date{\today}
\author{Tyler Foster}

\begin{document}

\maketitle

\begin{abstract}
   Lots of the structure of tonal harmony emerges from the representation theory of $\text{GL}_{1}(\CC)$. I here begin the project of composing music using tonal structures that emerge from the representation theory of other, higher-dimensional algebraic groups, such as $\text{SL}_{2}(\CC)$. This isn't just some theoretical exercise. As these notes should make clear, implementing these tonal structures in code will be very difficult to pull off without the super detailed outline of the general theory that appears below.
\end{abstract}

\tableofcontents

\begin{section}{Homogeneous polynomials and chords.}

\begin{subsection}{Decomposing complex numbers for better signal analysis}
Fix an element $z\in\CC$. Fixing a positive real {\em period} $P\in\RR_{>0}$ once and for all, we can decompose $z$ into its real and imaginary parts as
	\begin{equation}\label{complex decomp}
	z
	\ =\ 
	z(A,\theta)
	\ =\ 
	\text{log}(A)+i\tfrac{2\pi}{P}\theta,
	\end{equation}
for unique $0<A<\infty$ and $0\le \theta<P$. We refer to the positive real number $$\lambda:=\frac{1}{P}$$ as the {\em frequency} of $z(A,\theta)$. One reason for decomposing $z$ as in Equation \eqref{complex decomp} is that it gives the exponential of $z$ a form relevant to music. Indeed, from Equation \eqref{complex decomp} we get
	$$
	e^{z}
	\ =\ 
	A\ \!e^{i2\pi\lambda\theta},
	$$
with real and imaginary parts
	$$
	\text{Re}(e^{z})
	\ =\ 
	A\ \text{cos}(2\pi\lambda\theta)
	\ \ \ \ \ \ \ \ \ \text{and}\ \ \ \ \ \ \ \ \ 
	\text{Im}(e^{z})\ =\ A\ \text{sin}(2\pi\lambda\theta),
	$$
respectively. If we fix $A$ and let $\theta$ change linearly at the rate of $1$ unit per second, then both $\text{Re}(e^{z})$ and $\text{Im}(e^{z})$ describe a ``pure'' tone playing with amplitude $A$ at $\lambda\ \text{Hz}$,\footnote{\ Here $A$ is measured in units of $A_0 \ e^{20\text{dB}}$, where $A_{0}$ is some fixed reference amplitude.} such that the tone associated to $\text{Re}(e^{z})$ and the tone associated to $\text{Im}(e^{z})$ are out of phase by a quarter period $\frac{P}{4}$.

\end{subsection}

\begin{subsection}{Independent complex variables}
	[...]
	$$
	e^{z_1}
	=
	A_1\ e^{i2\pi\lambda\theta_1}
	\ \ \ \ \ \ \ \ \ \ \text{and}\ \ \ \ \ \ \ \ \ 
	e^{z_2}
	=
	A_2\ e^{i2\pi\lambda\theta_2}.
	$$
Note the possibility here of choosing two different values for our period, $P_1$ and $P_2$ say, to get
	$$
	e^{z_1}
	=
	A_1\ e^{i\tfrac{2\pi}{P_1}\theta_1}
	=
	A_1\ e^{i2\pi\lambda_1\theta_1}
	\ \ \ \ \ \ \ \ \ \ \text{and}\ \ \ \ \ \ \ \ \ 
	e^{z_2}
	=
	A_2\ e^{i\tfrac{2\pi}{P_2}\theta_2}
	=
	A_2\ e^{i2\pi\lambda_2\theta_2},
	$$
where $\lambda_1=\tfrac{1}{P_1}$ and $\lambda_2=\tfrac{1}{P_2}$. 

Given a function $f(x,y)$ of two variables, we can evaluate $f$ at $x=e^{z_1}$ and $y=e^{z_2}$ to obtain the value $f(e^{z_1},e^{z_2})$. In the special case that $f(x,y)$ is a {\em Laurent monomial}, i.e., that
	$$
	f(x,y)=x^my^n,
	$$
for $m,n\in\ZZ$, we have
	$$
	f(e^{z_1}, e^{z_2})
	\ =\ 
	A_1A_2\ e^{i2\pi (m\lambda_1\theta_1+n\lambda_2\theta_2)}
	\ =\ 
	A_1A_2\ e^{i2\pi\big(\tfrac{m}{P_1}\theta_1+\tfrac{n}{P_2}\theta_2\big)}.
	$$
The real and imaginary parts of this are
	\begin{equation}\label{FM style 1}
	\text{Re}\ f(e^{z_1}, e^{z_2})
	\ =\ 
	A_1A_2\ \text{cos}\Big(2\pi\big(\tfrac{m}{P_1}\theta_1+\tfrac{m}{P_2}\theta_2\big)\Big)
	\end{equation}
	$$
	\ \ \ \ \ \ \text{and}\ \ \ \ \ \ 
	$$
	\begin{equation}\label{FM style 2}
	\text{Im}\ f(e^{z_1}, e^{z_2})
	\ =\ 
	A_1A_2\ \text{sin}\Big(2\pi\big(\tfrac{m}{P_1}\theta_1+\tfrac{m}{P_2}\theta_2\big)\Big)
	.
	\end{equation}
This is a situation ripe for techniques from frequency modulation. For instance, if we let $t$ denote our time variable, in units of seconds, and we define
	$$
	\theta_1(t)\ =\ t
	\ \ \ \ \ \ \ \ \ \text{and}\ \ \ \ \ \ \ \ \ 
	\theta_2(t)\ =\ \text{sin}(\omega\ t)\ \ \ \text{for some}\ \omega\in\RR_{>0},
	$$
then the formulas in Equations \eqref{FM style 1} and \eqref{FM style 2} become instances of \href{https://en.wikipedia.org/wiki/Frequency_modulation_synthesis}{{\em FM synthesis}}. We can also see that the formulas in Equations \eqref{FM style 1} and \eqref{FM style 2} give us a broad generalization of FM synthesis, in that we can use any pair of functions
	$$
	\theta_1(t)\ \ \ \ \ \ \text{and}\ \ \ \ \ \ \ \theta_2(t)
	$$
of $t$ that we like. In this way, a kind of generalized FM synthesis realizes one version of the notion of ``pitch movement in $2$ dimensions.''  It is important to keep in mind here that the linear independence between the $2$ dimensions here is happening in the logarithm here. In other words, the linear independence is in the variable $z$, not in the value $e^{z}$.

This begins to move into the realm of harmony. We pause our development of harmony here, and pick it back up in {\color{red} \S[...]}

\end{subsection}



\end{section}

\vskip 1cm

\begin{section}{Representations of $\text{SL}_{2}(\CC)$.}

We let $\text{SL}_{2}(\CC)$ act on $\CC[x,y]$ through its inverse action on the argument $(x,y)$ of each function $f(x,y)\in\CC[x,y]$. Thus the matrix $\left(\begin{smallmatrix}a & b\\ c & d\end{smallmatrix}\right)\in\text{SL}_{2}(\CC)$ acts on each $f(x,y)\in\CC[x,y]$ via the action of its inverse $\left(\begin{smallmatrix}d & -b\ \\ -c\  & a\end{smallmatrix}\right)$

\end{section}


	







\vskip .75cm

\end{document}
